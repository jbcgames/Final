\documentclass{article}
\usepackage[utf8]{inputenc}
\usepackage[spanish]{babel}
\usepackage{listings}
\usepackage{graphicx}
\usepackage{cite}

\begin{document}

\begin{titlepage}
    \begin{center}
        \vspace*{1cm}
            
        \Huge
        \textbf{Proyecto Final}
            
        \vspace{0.5cm}
        \LARGE
        Los primeros pasos
            
        \vspace{1.5cm}
            
        \textbf{Miguel Ángel Alvarez Guzmán}
            
        \vfill
            
        \vspace{0.8cm}
            
        \Large
        Departamento de Ingeniería Electrónica y Telecomunicaciones\\
        Universidad de Antioquia\\
        Medellín\\
        Marzo de 2021
            
    \end{center}
\end{titlepage}

\tableofcontents
\newpage
\section{Introducción.}\label{intro}
El propósito de este documento es dar a conocer ideas de como se puede realizar el trabajo final de "Informática II". Siguiendo los parámetros dados en las clases teóricas y basando mis opiniones en la enseñanza de lo que se puede y como se puede hacer en el lenguaje c++, todo esto transmitido a través de un programa en la modalidad de "juego".
\section{Ideas Propuestas.} \label{contenido}
\subsection{Bases.}
El juego en si se trata de una persona que maneja una cámara de seguridad, la pantalla esta totalmente negra y abran varios ladrones detrás de ella, la cámara apuntara a un pequeño espacio y en ese espacio se vera una luz.
\subsubsection{Idea 1.}
La primera idea trata de que al momento de apuntar con la cámara (es el mouse) a un lugar de la pantalla y esta se alumbre mostrar lo que halla detrás de ella.
\subsubsection{Idea 2.}
La segunda idea trata de que al momento de ver la parte iluminada si hay un ladrón detrás y si se hace clic, el ladrón quede atrapado en una celda.
\subsubsection{Idea 3.}
La tercera idea es que los ladrones se deben estar moviendo por todo el escenario oscuro y si uno de ellos se choca con un ladrón capturado este lo liberara.
\subsubsection{Idea 4.}
La cuarta idea consiste en que el juego sera contra reloj, decir, hay un tiempo limite para que el jugador intente atrapar a todos los ladrones.
\subsubsection{Idea 5.}
La quinta y ultima idea consiste en que al pasar cada nivel el nivel de desplazamiento de los ladrones ira aumentando hasta llegar al nivel final, el jugador después de haber completado el juego podrá elegir el nivel de dificultad a su gusto.
\subsection{Equipo de trabajo.}
Este trabajo estará realizado solo por mi persona, lo cual implica una gran utilización de mi tiempo en el desarrollo de el código, elaboración de sprites y solución de errores, pero considero que tengo la capacidad y dedicación necesaria para llevar esto a cavo.

\section{Palabras Finales.}
Considero importante la realización de este proyecto en la elaboración de mi carrera, ya que me dará las bases necesarias para poder enfrentarme a futuro a problemas mas complejos y difíciles.

\end{document}
